\clearpage
\section*{\currfilename}

\begin{figure}[H]
  \fontsize{14pt}{14pt}\selectfont
  \begin{center}
    \begin{tikzpicture}[auto, scale=0.6, every node/.style={transform shape}, node distance=0.1cm, >=latex']
      \node[input](input00){};
      \node[input,right of=input00, node distance = 2.0cm](input01){};
      \node[input, above of=input01, node distance = 3.5cm](input1){};
      \node[gainblockright,right of=input1, node distance=4.0cm, minimum height=0.1cm, minimum width=1.0cm] (block1) {$K_{q_{\text{cmd}}}$};
      \node[whitesum,right of=block1, node distance=2.5cm] (sum1) {};
      \node[squareblock, minimum height=1cm, minimum width=2cm, right of=sum1,node distance=2.5cm] (block2) {\shortstack{Nominal \\ short period}};
      \node[output, right of=block2,node distance=2.5cm] (output1) {};
      \node[output, right of=output1,node distance=1.0cm] (output2) {};
      \node[output, right of=output2,node distance=2.5cm] (output3) {};
      \node[gainblockleft, above of=block2,node distance=2.0cm, minimum width=1.0cm](block3){\rotatebox{180}{$K_{\alpha}$}};
      \node[gainblockleft, above of=block3,node distance=2.0cm, minimum width=1.0cm](block4){\rotatebox{180}{$K_{q}$}};

      % Draw lines
      % \draw[->](input1) -- node[pos=0.4]{$q_{\text{cmd}}$} (block1);
      \draw[->](input1) -- (block1); % node[pos=0.4]{$r$}
      \draw[->](block1) -- (sum1);
      %\draw[->](sum1) -- node[pos=0.5]{$\delta_{e}$}(block2);
      \draw[->](sum1) -- (block2);
      % \draw[vecArrow](block2) -- node[pos=0.8]{$\alpha$, $q$}(output3);
      \draw[vecNoArrow](block2) -- node[pos=0.8]{$\alpha_{m},\; q_{m}$} (output3);
      \draw[->](output1) |- (block3);
      \draw[->](output2) |- (block4);
      \draw[->](block3) -| (sum1);
      \draw[->](block4) -| (sum1);

      \begin{pgfonlayer}{background}
        \path (block1 |- block1)+(-1.5,1.2) node (c) {};
        \path (block4 -| block4)+(4.0,-1.2) node (d) {};
        \path[fill=gray!40, draw, dashed] (c) rectangle (d);
      \end{pgfonlayer}

      \node[input,below of=input01, node distance=3.5cm](input1a){};
      \node[gainblockright,right of=input1a, node distance=4.0cm, minimum height=0.1cm, minimum width=1.0cm] (block1a) {$\Theta_{q_{\text{cmd}}}$};
      \node[whitesum,right of=block1a, node distance=2.5cm] (sum1a) {};
      \node[squareblock, minimum height=1cm, minimum width=2cm, right of=sum1a,node distance=2.5cm] (block2a) {\shortstack{Uncertain \\ short period}};
      \node[output, right of=block2a,node distance=2.5cm] (output1a) {};
      \node[output, right of=output1a,node distance=1.0cm] (output2a) {};
      \node[output, right of=output2a,node distance=2.5cm] (output3a) {};
      \node[gainblockleft, above of=block2a,node distance=2.0cm, minimum width=1.0cm](block3a){\rotatebox{180}{$\Theta_{\alpha}$}};
      \node[gainblockleft, above of=block3a,node distance=2.0cm, minimum width=1.0cm](block4a){\rotatebox{180}{$\Theta_{q}$}};

      % Draw lines
      \draw[->](input1a) -- (block1a); % node[pos=0.4]{$q_{\text{cmd}}$}
      \draw[->](block1a) -- (sum1a);
      \draw[->](sum1a) -- node[pos=0.5]{$\delta_{e}$}(block2a);
      % \draw[vecArrow](block2) -- node[pos=0.8]{$\alpha$, $q$}(output3);
      \draw[vecNoArrow](block2a) -- node[pos=0.8]{$\alpha,\; q$} (output3a);
      \draw[->](output1a) |- (block3a);
      \draw[->](output2a) |- (block4a);
      \draw[->](block3a) -| (sum1a);
      \draw[->](block4a) -| (sum1a);

      \begin{pgfonlayer}{background}
        \path (block1a |- block1a)+(-1.5,1.2) node (c) {};
        \path (block4a -| block4a)+(4.0,-1.2) node (d) {};
        \path[fill=gray!40, draw, dashed] (c) rectangle (d);
      \end{pgfonlayer}

      % Adaptive Arrow
      \coordinate (arrow1a) at ([xshift=-1.0cm,yshift=-1.0cm] block1a);
      \coordinate (arrow2a) at (block1a.225);
      \coordinate (arrow3a) at (block1a.45);
      \coordinate (arrow4a) at ([xshift=1.0cm,yshift=1.0cm] block1a);
      \draw[-](arrow1a) -- (arrow2a);
      \draw[->](arrow3a) -- (arrow4a);

      % Adaptive Arrow
      \coordinate (arrow1a) at ([xshift=-1.0cm,yshift=-1.0cm] block3a);
      \coordinate (arrow2a) at (block3a.225);
      \coordinate (arrow3a) at (block3a.45);
      \coordinate (arrow4a) at ([xshift=1.0cm,yshift=1.0cm] block3a);
      \draw[-](arrow1a) -- (arrow2a);
      \draw[->](arrow3a) -- (arrow4a);

      % Adaptive Arrow
      \coordinate (arrow1a) at ([xshift=-1.0cm,yshift=-1.0cm] block4a);
      \coordinate (arrow2a) at (block4a.225);
      \coordinate (arrow3a) at (block4a.45);
      \coordinate (arrow4a) at ([xshift=1.0cm,yshift=1.0cm] block4a);
      \draw[-](arrow1a) -- (arrow2a);
      \draw[->](arrow3a) -- (arrow4a);

      \node[gainblockleft, above of=block2a,node distance=2.0cm, minimum width=1.0cm](block3a){\rotatebox{180}{$\Theta_{\alpha}$}};
      \node[gainblockleft, above of=block3a,node distance=2.0cm, minimum width=1.0cm](block4a){\rotatebox{180}{$\Theta_{q}$}};

      \draw[-](input00) -- node[pos=0.4]{$q_{\text{cmd}}=r$} (input01);
      \draw[-](input01) -- (input1);
      \draw[-](input01) -- (input1a);

      \node[whitesum,below of=output3, node distance=3.5cm] (output6) {};
      \draw[vecArrow](output3) -- (output6);
      \draw[vecArrow](output3a) -- (output6);
      \node[output,right of=output6, node distance=2.5cm] (output7) {};
      \draw[vecArrow](output6) -- node[pos=0.3]{$e_{x}$} (output7);
    \end{tikzpicture}
  \end{center}
\end{figure}
