\clearpage
\section*{\currfilename}

\begin{tikzpicture}[x         = 1mm,
                    y         = 1mm,
                    >         = latex,
                    line join = round,
                    font      = \small]

  %%%%%%%%%%%%%%%%%%
  % NOZZLE DRAWING %
  %%%%%%%%%%%%%%%%%%
  %-> Origin definition
  \coordinate (o) at (0,0);

  %-> Nozzle
  % Symmetric characteristic is used in a
  % foreach command where the cycle is made
  % for upper part (up) with positive sign (+)
  % and lower part (down) with negative sign (-)
  \foreach \pos/\sign in {up/+,down/-}{

  %%-> Points definitions
  \coordinate (A\pos)        at ($(o)+(0,\sign11.5)$);
  \coordinate (B\pos)        at ($(A\pos)+(0,\sign5.075)$);
  \coordinate (C\pos)        at ($(B\pos)+(20,0)$);
  \coordinate (D\pos)        at ($(C\pos)+(0,-\sign1.575)$);
  \coordinate (E\pos)        at ($(D\pos)+(10,0)$);
  \coordinate (F\pos)        at ($(E\pos)+(0,\sign11)$);
  \coordinate (G\pos)        at ($(F\pos)+(0,\sign8)$);
  \coordinate (H\pos)        at ($(G\pos)+(0,\sign6)$);
  \coordinate (I\pos)        at ($(H\pos)+(10,0)$);
  \coordinate (L\pos)        at (G\pos-|I\pos);
  \coordinate (M\pos)        at (F\pos-|I\pos);
  \coordinate (N\pos)        at (E\pos-|I\pos);
  \coordinate (O\pos)        at ($(N\pos)+(100,0)$);
  \coordinate (P\pos)        at ($(O\pos)+(10,-\sign10)$);
  \coordinate (throat_\pos)  at ($(P\pos)-(43.8,\sign1.15)$);
  \coordinate (IN_\pos)      at ($(A\pos)+(92.74,0)$);
  \coordinate (center1_\pos) at ($(o)+(92.74,\sign4.5)$);
  \coordinate (center2_\pos) at ($(throat_\pos)+(0,\sign7)$);

  %%-> Draw nozzle main body
  \draw[fill,
        pattern    = section,
        line width = 1.1pt]
  \ifnum\sign1>0
          (center1_\pos)++(45:7)arc(45:90:7)--
  \else
          (center1_\pos)++(-45:7)arc(315:270:7)--
  \fi
  (A\pos)--
  (B\pos)--
  (C\pos)--
  (D\pos)--
  (E\pos)--
  (F\pos)--
  (G\pos)--
  (H\pos)--
  (I\pos)--
  (L\pos)--
  (M\pos)--
  (N\pos)--
  (O\pos)--
  (P\pos)--
  (throat_\pos)
  \ifnum\sign1>0
        arc(270:225:7)
  \else
       arc(90:135:7)
  \fi
  --cycle;

  %%-> Draw holes on the flange
  \draw[fill       = white,
        line width = 1.1pt] (G\pos)rectangle(M\pos);

  %%-> Draw symmetric axis on flange holes
  \draw[dash pattern = on 3pt off 5pt on 6pt off 5pt,
        line width   = 1pt] ($(F\pos)!.5!(G\pos)-(5,0)$)--
                            ($(L\pos)!.5!(M\pos)+(5,0)$);

  %%-> screw drawing
  \draw[dashed] (D\pos)--(D\pos-|A\pos);
  }

  %%-> Nozzle input and output closure
  \draw[line width = 1.1pt](Adown)--(Aup)
                           (Pdown)--(Pup);

  %%-> Nozzle symmetry line
  \draw[dash pattern = on 3pt off 5pt on 6pt off 5pt,
        line width   = 1pt] ($(Adown)!.5!(Aup)-(5,0)$)--($(Pdown)!.5!(Pup)+(5,0)$);

  %%%%%%%%%%%%%%
  % DIMENSIONS %
  %%%%%%%%%%%%%%
  % Macro to see the dimension
  % inserted. For debug.
  \newif\ifdimension
  \dimensionfalse
  %\dimensiontrue % if true, you will see the dimension number (%x) on the draw
  \newcount\Ndim=0
  \def\SeeDim#1{\ifdimension\global\advance\Ndim by 1 \the\Ndim\else#1\fi}
  %-> 1
  \Hdimension[text      = \SeeDim{10},
              distance  = 3]  (Hup) --  (Iup);

  %-> 2
  \Hdimension[text     = \SeeDim{10},
              distance = 3] (Cup)--(Hup);

  %-> 3
  \Hdimension[text     = \SeeDim{20},
              distance = 26.425] (Bup)--(Cup);

  %-> 4
  \Hdimension[text     = \SeeDim{100},
              distance = 3] (Iup)--(Oup);

  %-> 5
  \Hdimension[text     = \SeeDim{10},
              distance = 28] (Oup)--(Pup);

  %-> 6
  \dimension[text     = \SeeDim{\diameter30},
             distance = -25] (Odown)--(Oup);

  %-> 7
  \Hdimension[text     = \SeeDim{43.8},
              distance = -48.15] (throat_down)--(Odown);

  %-> 8
  \Hdimension[text     = \SeeDim{92.74},
              distance = -40.5] (IN_down)--(Bdown);

  %-> 9
  \dimension[text             = \SeeDim{\diameter7.7},
             text translation = -7mm,
             distance         = 20] (throat_down)--(throat_up);

  %-> 10
  \dimension[text     = \SeeDim{\diameter10},
             distance = -8] (Pdown)--(Pup);

  %-> 11 (by hand)
  \draw[->] (center1_up)--++(45:7)node[sloped,
                                       above     = .4mm,
                                       pos       = .3,
                                       inner sep = 0.5pt]{\SeeDim{R7}};

  %-> 12 (by hand)
  \draw[->] (center2_up)--++(225:7)node[above     = .4mm,
                                        pos       = .4,
                                        inner sep = 0.5pt,
                                        rotate    = 45,
                                        fill      = white]{\SeeDim{R7}};

  %-> 13 (by hand)
  \coordinate (raccordo_up)   at ($(center1_up)+(45:7)$);
  \coordinate (raccordo_down) at ($(center1_down)+(-45:7)$);
  \draw (raccordo_up)--++(135:20);
  \draw (raccordo_down)--++(225:20);
  \path[name path=C1](raccordo_up)--++(-45:20);
  \path[name path=C2](raccordo_down)--++(45:20);
  \path[name intersections={of=C1 and C2}];
  \coordinate (C90) at (intersection-1);
  \draw[<->] ($(C90)+(135:30)$)arc[start angle = 135,
                                   delta angle = 90,
                                   radius      = 30];
  \def\angle{35}
  \node[rotate = \angle+45,
        anchor = south] at ($(C90)+(135+\angle:30)$){\SeeDim{\ang{90}}};

  %-> 14
  \dimension[text             = \SeeDim{\diameter60},
             distance         = -5,
             >                = angle 45,
             text translation = .5cm] ($(Ldown)!.5!(Mdown)$)--($(Lup)!.5!(Mup)$);

  %-> 15
  \dimension[text             = \SeeDim{\diameter80},
             distance         = -10,
             text translation = .5cm] (Idown)--(Iup);

  %-> 16
  \dimension[text             = \SeeDim{\diameter8 ($\times$4)},
             distance         = 10,
             text translation = -12mm] (Gdown)--(Fdown);

  %-> 17
  \dimension[text     = \SeeDim{G1'},
             distance = 10] (Bdown)--(Bup);

  %-> 18
  \dimension[text     = \SeeDim{\diameter23},
             distance = 5] (Adown)--(Aup);

  % Dimensions scale
  \node[anchor = north west] at (current bounding box.south west)
  {All dimensions are in millimeters};
\end{tikzpicture}
