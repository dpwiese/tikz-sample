\clearpage
\section*{\currfilename}

% We need layers to draw the block diagram
\pgfdeclarelayer{background}
\pgfdeclarelayer{foreground}
\pgfsetlayers{background,main,foreground}

\begin{figure}[H]
  \fontsize{14pt}{14pt}\selectfont
  \begin{center}
    \begin{tikzpicture}[auto, scale=0.6, every node/.style={transform shape}, node distance=0.1cm, >=latex']
      \node[input](input1){};
      \node[gainblockright,right of=input1, node distance=4.0cm, minimum height=0.1cm, minimum width=1.0cm] (block1) {$K_{q_{\text{cmd}}}$};
      \node[whitesum,right of=block1, node distance=2.5cm] (sum1) {};
      \node[squareblock, minimum height=1cm, minimum width=2cm, right of=sum1,node distance=2.5cm] (block2) {\shortstack{Short \\ Period}};
      \node[output, right of=block2,node distance=2.5cm] (output1) {};
      \node[output, right of=output1,node distance=1.0cm] (output2) {};
      \node[output, right of=output2,node distance=2.5cm] (output3) {};
      \node[gainblockleft, above of=block2,node distance=2.0cm, minimum width=1.0cm](block3){\rotatebox{180}{$K_{\alpha}$}};
      \node[gainblockleft, above of=block3,node distance=2.0cm, minimum width=1.0cm](block4){\rotatebox{180}{$K_{q}$}};

      % Draw lines
      % \draw[->](input1) -- node[pos=0.4]{$q_{\text{cmd}}$} (block1);
      \draw[->](input1) -- node[pos=0.4]{$r$} (block1);
      \draw[->](block1) -- (sum1);
      %\draw[->](sum1) -- node[pos=0.5]{$\delta_{e}$}(block2);
      \draw[->](sum1) -- (block2);
      % \draw[vecArrow](block2) -- node[pos=0.8]{$\alpha$, $q$}(output3);
      \draw[vecArrow](block2) -- (output3);
      \draw[->](output1) |- (block3);
      \draw[->](output2) |- (block4);
      \draw[->](block3) -| (sum1);
      \draw[->](block4) -| (sum1);

      \begin{pgfonlayer}{background}
        \path (block1 |- block1)+(-1.5,1.2) node (c) {};
        \path (block4 -| block4)+(4.0,-1.2) node (d) {};
        \path[fill=gray!40, draw, dashed] (c) rectangle (d);
      \end{pgfonlayer}
    \end{tikzpicture}
  \end{center}
\end{figure}
